\newglossaryentry{elasticsearch}
{name={Elasticsearch}, description={Base de données de type NoSQL orienté document. Elasticsearch est un moteur de recherche libre open source utilisant Lucene.}}

\newglossaryentry{sprint}
{name={sprint},description={Le sprint est une période d'un mois au maximum, au bout de laquelle l'équipe délivre un incrément du produit, potentiellement livrable. Une fois la durée choisie, elle reste constante pendant toute la durée du développement. Un nouveau sprint démarre dès la fin du précédent. Chaque sprint possède un but et on lui associe une liste d'éléments du carnet du produit (fonctionnalités) à réaliser. (Voir partie sur l'agilité).}}

\newglossaryentry{wiki}
{name={wiki},description={Un wiki est une application web qui permet la création, la modification et l'illustration collaboratives de pages à l'intérieur d'un site web. Il utilise un langage de balisage et son contenu est modifiable au moyen d’un navigateur web. C'est un outil de gestion de contenu, dont la structure implicite est minimale, tandis que la structure explicite émerge en fonction des besoins des usagers.}}

\newglossaryentry{newsletter}
{name={newsletter}, description={Une lettre d'information, newsletter ou infolettre est un document d'information envoyé de manière périodique par courrier électronique à une liste de diffusion regroupant l'ensemble des personnes qui y sont inscrites. Une lettre d'information peut également être téléchargée depuis un site web.}}

\newglossaryentry{templates}
{name={template},description={
        Modèle de conception de logiciel ou de présentation des données.
        Synonymes 
        \begin{itemize}
        \item Patron
        \item Modèle
        \item Gabarit
        \end{itemize}}
        }

\newglossaryentry{IDE}
{name={IDE}, description={En programmation informatique, un environnement de développement est un ensemble d'outils pour augmenter la productivité des programmeurs qui développent des logiciels. Il comporte un éditeur de texte destiné à la programmation, des fonctions qui permettent, par pression sur un bouton, de démarrer le compilateur ou l'éditeur de liens ainsi qu'un débogueur en ligne, qui permet d'exécuter ligne par ligne le programme en cours de construction. Certains environnements sont dédiés à un langage de programmation en particulier.}}

\newglossaryentry{vache}{
name={vache à lait},
description={Une vache à lait a part de marché relative élevée sur un marché en faible croissance, en phase de maturité ou en déclin. 
Exigeant peu d'investissements nouveaux et dégageant des flux financiers importants qui devront être réinvesti intelligemment sur les vedettes et les dilemmes}
}


\newglossaryentry{part de marche en valeur}
{name={part de marché en valeur}, description={La PDM valeur ou part de marché valeur désigne la part de marché d’un produit exprimée en valeur monétaire et non en quantité.}}


\newglossaryentry{depot}
{name={dépôt git}, description={Les dépôts distants sont des versions de votre projet qui sont hébergées sur Internet ou le réseau. Vous pouvez en avoir plusieurs, pour lesquels vous pouvez avoir des droits soit en lecture seule, soit en lecture/écriture. Collaborer avec d'autres personnes consiste à gérer ces dépôts distants, en poussant ou tirant des données depuis et vers ces dépôts quand vous souhaitez partager votre travail.}}


\newglossaryentry{commit}
{name={commit},description={Le terme anglais commit désigne une validation de transaction qui fait référence à la commande synonyme Commit présente dans la plupart des systèmes de gestion de base de données et des logiciels de gestion de versions. Dans les systèmes de bases de données et de révision de fichier, la validation est une exécution de la tâche préalablement confiée, marquant à la fois la fin de la demande de transaction et le début de l’exécution de la tâche confiée, qui devra être exécutée atomiquement.}}

