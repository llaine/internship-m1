
\chapter{Summary}

\section{Français}

Au cours du stage que j’ai réalisé au sein de la société Bob El Web, j’ai été chargé de travailler au développement de nouvelles fonctionnalités sur le logiciel Purple Base à partir du logiciel déjà existant, Bob Booking.

Pour apporter des éléments de réponse aux questions de l'équipe technique, en particulier pour
proposer la mise en place de nouvelles solutions de stockage et de traitement de données mettre en place un outillage de test développer un outil de création d'emailing répondre à un cahier des charges.  
J’ai du accumuler un certain nombre de connaissances et rechercher les "bonne pratiques" à mettre en place.
 
Confronté aux difficultés de conception d'un logiciel, je me suis concentré sur les bonnes pratiques à respecter, du début d’un projet et tout au long de son développement. 
Cette expérience de développement d'un nouvel applicatif à partir d’un l'applicatif existant, m'a amené à porter ma réflexion sur plusieurs sujets :

\begin{itemize}
\item Quelle architecture utiliser pour un nouveau projet ? 
\item Comment penser sa maintenance à long terme ? 
\item Comment anticiper les possibilités d'évolutions? 
\end{itemize}

Pour surmonter ces difficultés, les nombreux échanges et la concertation avec l'équipe technique de Bob Booking ont été précieux. 
Ils m’ont permis de prendre en compte leurs expériences antérieures de développeurs et de réfléchir sur des concepts, démarches et outils permettant de concevoir un projet informatique sur de meilleures bases. 

Dans mon rapport de stage je vais chercher à montrer comment la réflexion centrée sur ces concepts m'a aidé à trouver de meilleures pratiques d'un point de vue technique et fonctionnel sur le développement d'un projet informatique, dans le but d'obtenir un logiciel de qualité maintenable et évolutif

Dans le chapitre 3, je présenterais l'entreprise et son activité. 

Le chapitre 5, sera consacré au pilotage de la qualité d'un point de vue "technique" en abordant les problèmes liés à une négligence de la qualité, le concept de dette technique et les solutions à suivre pour minimiser la dette technique sur les projets.

Le chapitre 6, sera consacré à montrer comment l'utilisation de la méthode Agile dans la gestion de projet m'a permit d'apprendre à mettre en oeuvre l'ensemble des pratiques abordés dans le chapitre 5.
Comment à l'aide de cette gestion de projet, il est facile d'intégrer ces méthodes durant le cycle de développement du logiciel et donc d'avoir un pilotage de la qualité vraiment efficace.


\section{Anglais}

During my internship in Bob el Web, I had to work on the new product called "Purple Base", on which I had to develop new features. 

To provide anwsers for the technical team, especially about storage solutions, data processing or set up a test workflow for the Purple Base application, develop an emailing tool. 

I had to accumulate a certain amout of knowledge and look for "good practices" to implement. 

Faced with software design challenges, I focused on good practices to work with from the beginning of the project and throughouts its development. 

This development experience brought me to think about several topics : 

\begin{itemize}
\item What architecture to use for a new project ? 
\item How to think about the long-term maintenance for the project ?
\item How to anticipate the possibilities of changes ?
\end{itemize}

To overcome theses difficulties, the exchanges and cooperation with the technical team was very helpful. 

They shared with me their developer experience which allowed me to reflect on concepts, approaches and tools for designing IT project on a better footing. 


In this report, I will try to show you how the reflection centered on these concepts, helped me to find best practices from a technical and a functional point of view on the development of IT projects, in order to obtain a maintanable and scalable software. 

Chapter 3, will introduce the company and the business. 

Chapter 5, will be devoted to quality on a technical point of view, talking about issues caused by a lack of quality in software development, the concept of technical debt and the solutions to follow in order to minimize technical debt on projects. 

Chapter 6, will be dedicated to show, how the use of Agile in project management allowed me to learn how to implement all the practacices discussed in chapter 5. 
By using this type of project management, it is easy to integrate these methods in the software development cycle and have an effective quality control. 