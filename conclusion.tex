\chapter{Conclusion}


Ce stage de fin de Master 1, m'a permis de travailler sur plusieurs missions.

Il était nécessaire d'apprendre le périmètre fonctionnel qui m'était confié et de ne pas seulement "foncer dans le code".
Prendre de la hauteur par rapport à ce qu'il m'était demandé, inscrire mon travail et ma réflexion dans un processus qualitatif, où la production du code est un maillon du processus et non le processus entier.

\section{Variétés des missions}
Les missions sur lesquelles j'ai été amené à travailler, étaient très intéressantes et m'ont permis d'aborder différents aspects du développement logiciel. 

Lorsqu'il a été question de réfléchir à une solution de stockage de données dans le cadre de la mise en place du référentiel Purple Base, la quantité de code que j'ai écrite était moins importante que le  temps passé à réfléchir sur les problèmes d'architecture.
A l'aide de \textbf{Malo Pichot} l'administrateur de base de données et sysadmin de l'entreprise, 
j'ai réfléchi sur les différentes techniques permettant de stocker les données, les différents schémas à adopter et les moyens de gérer la base de données sur un serveur. % Notions sysadmin/dba 

% Dialogue/communication
Egalement, lorsqu'il a fallu réfléchir à la mise en place de l'outillage de test. 
Avec \textbf{Maxime Sénécat} mon maitre de stage, nous avons travaillé en binôme, dialogués et débattus des différentes solutions, afin d'en sélectionner une fiable, efficace et évolutive.  

% Code, dialogue prestataire
Lorsque j'ai travaillé sur l'outil de newsletter, j'ai du réfléchir à des notions d'architecture logiciel, de qualité mais aussi du dialoguer avec le patron, l'équipe technique afin de réaliser le cahier des charges et le chiffrage. 
Puis j'ai traité avec les différents prestataires (graphiste, etc) du projet afin de produire une solution optimale.

Enfin toutes ces missions, ont été réalisés à l'aide d'une gestion de projet Agile et en particulier le scrum.
Cela a été l'occasion d'apprendre beaucoup sur cette nouvelle méthode que j'ai pu appliquer directement aux travaux que j'ai du réaliser. 

\section{Esprit d'initiative et processus moins figé}

Ce stage de fin de M1, a également été pour moi l'occasion de faire un constat la différence existant entre les petites entreprises et les grosses structures (SSII, grosses entreprises, etc).

Les grosses entreprises, ont tendances à formaliser et standardiser les processus de création de développement logiciel, laissant moins de liberté et d'innovation pour les éxecutants (les développeurs). 

Le fait d'avoir effectué mon stage dans une petite structure, m'a permis de constater que le processus était beaucoup plus flexible, que chacun pouvait devenir force de proposition.

J'ai pu proposer des suggestions, des améliorations sur pleins de sujets différents.
Que ce soit les missions qu'on m'a confié mais aussi sur d'autres sujets de l'entreprise. 

Prendre part au collectif, être dans une logique d'échange avec les collègues plus experimentés permet d'apprendre beaucoup à partir des problèmes rencontrés. 


\section{Une expérience réussie}
Enfin d'un point de vue global, ce stage est pour moi une expérience réussie. 

Sur le plan professionnel, ça m'a permit d'appliquer les concepts étudiés en cours dans un cadre réel. 
J'ai également pu élargir mes connaissances en travaillant sur des sujets nouveaux, telle que la gestion de projet agile ou la mise en place de la qualité.

Enfin ça m'a conforté dans mon projet professionnel, m'encourageant à continuer dans l'informatique, qui me passionne chaque jour. 

Sur le plan personnel, c'était une expérience humaine très enrichissante et j'ai beaucoup apprécié toute l'équipe de Bob. 







