\usefont{OT1}{phv}{m}{n}

\chapter{Missions}

\section{Commande initiale}

C'est dans le contexte de la nouvelle application Purple Base que j'ai été recruté en stage. 
Ce logiciel devant être une application multi-plateforme j'étais en charge de développer la version tablette. 

Mais les fonctionnalités devant être adaptées sur tablette n'étant pas prête le jour de mon arrivé, j'ai donc travaillé sur d'autres missions tout au long de mon stage. 


\section{Missions réalisées durant le stage}

\subsection{Mise en place du référentiel}

% Pourquoi ?
Ma première mission fut de réfléchir à la mise en place d'une solution de stockage et de traitement de données dans le cadre du référentiel de données pour Purple Base. 
Une base de données avait déjà été selectionnée par l'équipe de développeurs appelée \gls{elasticsearch} 
pour ses capacités à pouvoir traiter et stocker de très grosses masses de données. 

% Qu'est-ce qui a été fait ? 
Pendant un \gls{sprint} je me suis plongé dans diverses ressources, avec comme but de collecter un maximum de connaissances sur \gls{elasticsearch}. 

Ce fut l'occasion de lire un livre sur le fonctionnement détaillé de la base de données, mais également de parcourir blogs et site diverses sur le sujet.

Puisque le fruit de mes recherches est destiné aux autres développeurs, il était primordial qu'une restitution écrite soit fournie, en plus de celle orale à la fin du sprint.
J'ai par conséquent documenté tout ce le travail effectué, sur le \gls{wiki} pour permettre à l'équipe d'avoir une trace du travail réalisé.

% Solutions trouvés
Cette phase de recherche ma permis de pouvoir accumuler des connaissances pour pouvoir répondre aux questions de l'équipe technique. 
J'ai également du me concentrer sur l'aspect "bonne pratiques" à mettre en place pour proposer une base solide et évolutive de la base données.  


\subsection{Outillage de test}
La seconde mission qui m'a été confiée fût de réfléchir à la mise en place de l'outillage de test sur Purple Base. 

% Pourquoi ? 
Pour que Purple Base soit vraiment le renouveau de Bob el Web, le pilotage de la qualité ne doit pas être mis à part. 
En effet, en ne négligeant pas cet aspect, le logiciel aura des bases solides et pourra être maintenable sur le long terme. 
\jumpOne
% Qu'est-ce qui a été fait ? 
Plusieurs sprints ont été consacrés à la recherche d'une solution pour mettre en place un processus de test fiable et évolutif dans le temps. 
Après avoir étudié plusieurs framework de test, j'ai sélectionné à l'aide de mon maitre de stage, Karma et Jasmine (deux framework spécialisé dans le test JavaScript, AngularJS), pour leur flexibilité et facilité d'utilisation. 

% Solutions trouvés et concepts abordés
Cette seconde mission a encore une fois été l'occasion d'alterner phases de recherche et phase de rédaction sur le wiki. 
Ce fut l'occasion pour moi, de me concentrer sur les bonnes pratiques du test à mettre en place sur des applications utilisant la technologie JavaScript/AngularJS. 
% Annexes screenshot wiki et bonnes pratiques. 


\subsection{Newsletter}

% Pourquoi ?

Dans le cadre de l'application Purple Base et de ses nouvelles fonctionnalités, j'ai dû travailler sur le développement d'un outil de création d'emailing. 

En août 2014, la société à réalisé un sondage sur les clients, pour connaitre quelles étaient les fonctionnalités les plus demandés dans Bob Booking. 
La possibilité de pouvoir créer et deployer par le biais d'un outil, des campagnes de communication numérique est apparu dans les premières position.
 

\subsubsection{Qu'est-ce qu'un outil d'emailing ?}
Ce logiciel, permet de créer des \gls{newsletter}, à l'aide de \gls{templates}.
Il permet d'automatiser l'envoi de mails, d'intégrer les contenus sur les réseaux sociaux mais aussi de gérer des fichiers de contacts et de faire du suivis statistique des campagnes envoyées. 

\subsubsection{Pourquoi l'e-mailing?}
C'est un moyen très simple et peu cher de faire connaitre un produit ou un projet sur internet.
Cette méthode consiste à envoyer des emails à des personnes, leur proposant le dit produit ou le projet. 
Les emails, vont permettre de très facilement cibler la population souhaitée qui serait le plus à même d'être touchée par notre produit. 

\paragraph{Exemple} Je vends un concert de Rock d'un groupe des années 80, j'aimerai donc envoyer des emails à mes clients, dans une tranche d'âge spécifique appreciant ce style musical. 

Il est évident que l'on ne va pas rédiger chaque email à la main. 
De même pour la création de l'email qui demande beaucoup de temps des compétences en design pour pouvoir réaliser quelque chose d'esthetique et professionnel. 

C'est là où l'outil de création rentre en jeu !

A partir de template pré-construit, on va pouvoir créer notre mail, ajoutant des images du texte, des liens, etc. 
% Annnexe : newsletter BOB
Cet email va pouvoir par la suite être envoyé à autant de personne souhaité (10 000, 20 000, etc) très facilement par le biais d'un système interne. 

% Qu'est-ce qui a été fait ? 
\subsubsection{Le projet}
L'entreprise avait prévu d'internaliser le développement de la partie gestion des envois, mais le créateur d'email lui, de par sa complexité a initialement été externalisé.

On m'a donc confié la réalisation du chiffrage, la rédaction du cahier des charges (écrit avec le patron de l'entreprise) % annexe : Mettre en annexe, cahier des charges et chiffrage. 
et rechercher un prestataire pour réaliser l'outil. 

Plusieurs prestataires ont répondu à l'appel, cependant soit leur prix ou leur disponibilité ne convenaient pas à l'entreprise. 
Etant à quelques semaines de la fin du stage, j'ai proposé l'idée que je pouvais moi même développer l'outil.
Cette idée a par la suite été acceptée.
J'allais pouvoir m'appliquer à ne pas négliger le coté pilotage de la qualité, étant donné que je passais du rôle de client à prestataire. 

% Concepts abordés
Cette dernière mission m'a vraiment permis de me concentrer sur des questions de bonnes pratiques de la qualité à mettre en place dès le début du développement et celles qu'il est nécessaire de suivre par la suite tout au long du projet. 
C'est aussi celle qui m'a permis de dégager la problématique liée à mon stage. 

\section{Problématique liés aux missions}
C'est donc tout naturellement que j'ai décidé d'orienter ma problèmatique vers le pilotage de la qualité. 
\jumpOne
Ces expériences de développement d'un nouvel applicatif à partir de l'existant (Bob Booking développé par la société en 2007), m'a confronté aux difficultés de conception d'un logiciel.
J'ai du porter ma réflexion sur plusieurs sujets tels que : 
\begin{itemize}
\item Quelle architecture utiliser pour un nouveau projet ? 
\item Comment penser sa maintenance à long terme ? 
\item Comment anticiper les possibilités d'évolutions 
\end{itemize}
Pour surmonter ces difficultés, la communication avec l'équipe technique ma permis de réfléchir en prenant en compte leurs expériences antérieurs de développeurs avec ses échec et ses réussites. 



